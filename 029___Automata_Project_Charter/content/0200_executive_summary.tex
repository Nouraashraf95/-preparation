% filepath: c:\Users\RAW\OneDrive\Documents\GitHub\-preparation\029___Automata_Project_Charter\content\0200_executive_summary.tex
\section{Partnership Executive Summary}
{%
\sffamily % Use sans-serif font for this section's text (scoped)

This project charter presents the implementation approach for a customized Odoo ERP system for Automata System Integration Solution. Its core objective is to strengthen Automata’s operational performance and efficiency. The document also serves as the formal framework for an ongoing digital transformation partnership between Automata and Scando Integrated Solutions. Unlike conventional projects with a fixed scope, this collaboration follows a subscription-based model that supports an unlimited number of evolving initiatives aligned with Automata’s business growth and changing requirements.

This document details the following critical areas for the project:

\begin{enumerate}[label=\arabic*., leftmargin=*, align=left, itemsep=0.8em]

    \item \textbf{The Subscription Model \& Initiative Framework}
    \begin{itemize}[label=\textbullet, itemsep=0.3em, leftmargin=*]
        \item The project plan is no longer a linear sequence of phases but a dynamic lifecycle applied to every new initiative.
        \item \textbf{Continuous Value Delivery:} Work is organized into discrete ``Initiatives,'' each treated as a mini-project with its own specialized scope.
    \end{itemize}

    \textbf{The Initiative Lifecycle:}
    \begin{itemize}[label=\textbullet, itemsep=0.2em, leftmargin=*]
        \item \textbf{Backlog Refinement:} Identifying the next business pain point.
        \item \textbf{Definition:} Setting specific scope, stakeholders, and success criteria for the initiative.
        \item \textbf{Sprints:} Iterative development and configuration cycles.
        \item \textbf{UAT \& Release:} Immediate deployment of value to production.
    \end{itemize}

    \begin{itemize}[label=\textbullet, itemsep=0.3em, leftmargin=*]
        \item \textbf{Scalability:} New modules or process improvements can be initiated at any time within the subscription period.
    \end{itemize}

    \item \textbf{Organizational Context}
    \begin{itemize}[label=\textbullet, itemsep=0.2em, leftmargin=*]
        \item Identify key stakeholders from Automata, Scando Integrated Solutions, and Odoo, detailing their roles and responsibilities.
        \item Outline the project's governance structure, including the Project Sponsor, Steering Committee, and Project Management Team.
        \item Acknowledge the importance of change management to ensure smooth adoption of the new system and processes by Automata staff.
    \end{itemize}

    \item \textbf{Project Plan and Approach}
    \begin{itemize}[label=\textbullet, itemsep=0.2em, leftmargin=*]
        \item Present a phased implementation approach, from initial planning and requirements gathering through to deployment, go-live, and post-implementation monitoring.
        \item Provide a high-level project timeline with key milestones for each phase.
        \item Recognize the resources required from both Automata and Scando Integrated Solutions for successful project execution.
        \item Briefly touch upon risk considerations and the need for ongoing management.
    \end{itemize}

    \item \textbf{Subscription Operating Model (Capacity and Activation)}
    \begin{itemize}[label=\textbullet, itemsep=0.2em, leftmargin=*]
        \item The organization may activate up to \textbf{two (2)} business lines concurrently at any given time.
        \item Each active business line may be supported by up to \textbf{three (3)} named Scando resources, as required by the initiative(s) in progress.
        \item The subscription provides delivery capacity of up to \textbf{8 hours per week per active business line}.
        \item Accordingly, the total weekly capacity is up to \textbf{16 hours per week} when two business lines are active.
        \item Full-year objectives and initiative candidates will be defined and confirmed during \textbf{Q1} as part of the strategic planning cycle. The sequencing and recommended execution order will be proposed by the Strategist and agreed with the organization through the governance process defined in this charter.
    \end{itemize}

\end{enumerate}

The success of this Odoo ERP implementation will be measured by its ability to provide Automata with a robust platform that supports its growth, improves operational efficiency, ensures financial accountability, and empowers its leadership with timely insights for strategic decision-making. This charter serves as the foundational agreement for all project stakeholders.

} % end scoped \sffamily
\newpage